\documentclass[11pt,a4paper]{report} 
\usepackage[portuges]{babel}
\usepackage[utf8]{inputenc}
\usepackage{graphicx}
\usepackage{amsthm}
\usepackage{amsfonts}
\usepackage{indentfirst}
\newtheorem{lema}{letter}
\newtheorem{corolario}{Corolario}
\newtheorem{teorema}{Teorema}
\renewcommand\thesection{\arabic{section}}
\graphicspath{{images//}}


\title{Soluções do livro "Curso de Análise" vol. 1}
\author{Ronald A. Kaiser}


\begin{document}
    %\maketitle
    %\newpage
    
    \section{Capítulo 1}
    \subsection{Questão 1}
    \begin{proof}
    Para demonstrarmos que $X$ = $A \cup B$, provaremos que \textbf{1.} $X \subset (A \cup B)$ e \textbf{2.} $(A \cup B) \subset X$.

    \paragraph{1.}
    Sabemos que $\forall A$ e $\forall B$, $A \subset (A \cup B)$ e $B \subset (A \cup B)$. 
    Da 2$^{a}$ hipótese, $A \subset (A \cup B)$ e $B \subset (A \cup B) \Rightarrow X \subset (A \cup B)$.

    \paragraph{2.}
    Da 1$^{a}$ hipótese, $x \in A \Rightarrow x \in X$ e $y \in B \Rightarrow y \in X$. Assim, todo elemento de A ou de B também pertence a X. Mais formalmente: $z \in A$ ou $z \in B \Rightarrow z \in X$. Portanto, ($A \cup B) \subset X$.

    \paragraph{}
    De \textbf{1.} e \textbf{2.}, $X$ = $(A \cup B)$.
    \end{proof}

        
\end{document}


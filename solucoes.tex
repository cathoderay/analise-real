\documentclass[9pt,twocolumn,a4paper]{article} 
\usepackage[portuges]{babel}
\usepackage[utf8]{inputenc}
\usepackage{graphicx}
\usepackage{amsthm,amssymb}
\usepackage{amsfonts}
\usepackage{indentfirst}
\newtheorem{lema}{letter}
\newtheorem{corolario}{Corolario}
\newtheorem{teorema}{Teorema}
\renewcommand\thesection{\arabic{section}}
\renewcommand{\qedsymbol}{\rule{0.7em}{0.7em}}
\graphicspath{{images//}}


\title{Soluções do livro "Curso de Análise" vol. 1}
\author{Ronald A. Kaiser}


\begin{document}
    \maketitle
    \newpage
    
    \paragraph{Notas}
    As soluções aqui apresentadas não contém seus respectivos enunciados. Algumas hipóteses fornecidas nos enunciados não são reproduzidas aqui, mas são utilizadas em algumas demonstrações. O leitor atento não deve ter dificuldades para acompanhar as soluções. Caso tenha alguma dúvida, recorra à obra original "Curso de análise" (vol. 1) de Elon Lages Lima.

    \section{Capítulo 1}

    \subsection{Questão 1}
    \begin{proof}
    Para demonstrarmos que $X = A \cup B$, provaremos que \textbf{1.} $X \subset (A \cup B)$ e \textbf{2.} $(A \cup B) \subset X$.

    \paragraph{1.}
    Sabemos que $\forall A$ e $\forall B$, $A \subset (A \cup B)$ e $B \subset (A \cup B)$. 
    Da 2$^a$ hipótese, $A \subset (A \cup B)$ e $B \subset (A \cup B) \Rightarrow X \subset (A \cup B)$.

    \paragraph{2.}
    A partir da 1$^a$ hipótese e da definição de inclusão, $x \in A \Rightarrow x \in X$ e $y \in B \Rightarrow y \in X$. Assim, todo elemento de A ou de B também pertence a X. Mais formalmente: $z \in A$ ou $z \in B \Rightarrow z \in X$, e  pela definição de união, $z \in (A \cup B) \Rightarrow z \in X$. Portanto, $(A \cup B) \subset X$.

    \paragraph{}
    De \textbf{1.} e \textbf{2.}, $X = (A \cup B)$.
    \end{proof}


    \subsection{Questão 2}
    \paragraph{Enunciado}
    Dados os conjuntos A e B, seja X um conjunto com as seguintes propriedades:

    1$^{a}$ $X \subset A$ e $X \subset B$

    2$^{a}$ $Y \subset A$ e $Y \subset B \Rightarrow Y \subset X$
    
    Prove que $X = A \cap B$.
    \begin{proof}
    Para demonstrarmos que $X = A \cap B$, provaremos que \textbf{1.} $X \subset (A \cap B)$ e \textbf{2.} $(A \cap B) \subset X$.

    \paragraph{1.}
    A partir da 1$^a$ hipótese e da definição de inclusão, segue que: $x \in X \Rightarrow x \in A$ e $x \in X \Rightarrow x \in B$. Assim, todo elemento de X pertence também aos conjuntos A e B. Portanto, $x \in X \Rightarrow x \in A$ e $x \in B$. Logo, pela definição de interseção, $x \in X \Rightarrow x \in (A\cap B)$. Donde concluímos que $X \subset (A \cap B)$.

    \paragraph{2.}
    Sabemos que $\forall A$ e $\forall B$, $(A \cap B) \subset A$ e $(A \cap B) \subset B$. Da 2$^{a}$ hipótese, temos que $(A \cap B) \subset A$ e $(A \cap B) \subset B \Rightarrow (A \cap B) \subset X$. Portanto, $(A \cap B) \subset X$.

    \paragraph{}
    De \textbf{1.} e \textbf{2.}, $X = (A \cap B)$.
    \end{proof}


    \subsection{Questão 3}
    Provaremos inicialmente que $A \cap B = \emptyset \Leftrightarrow A \subset B^c$.

    \begin{proof}
    Divideremos a prova em duas partes: \textbf{1.} $A \cap B = \emptyset \Rightarrow A \subset B^c$ e \textbf{2.} $ A\subset B^c \Rightarrow A \cap B = \emptyset$.

    \paragraph{1. $\Rightarrow)$}
    Suponhamos, por absurdo, que $A \not\subset B^c$. Neste caso, $\exists x | x \in A$ e $x \not\in B^c$. Mas, pela definição de complementar, $x \not\in B^c \Leftrightarrow x \in B$. Assim, $\exists x | x \in A$ e $x \in B \Leftrightarrow \exists x | x \in (A \cap B)$. Mas, por hipótese, $A \cap B = \emptyset$. Não pode existir tal $x$. Um absurdo gerado pela nossa suposição inicial. Portanto, $A \cap B = \emptyset \Rightarrow A \subset B^c$.


    \paragraph{2. $\Leftarrow)$}
    Suponhamos, por absurdo, que $A \cap B \not= \emptyset.$ Neste caso, $\exists x | x \in (A \cap B)$, e pela definição de interseção, $\exists x| x \in A $ e $x \in B$. Mas, $x \in B \Leftrightarrow x \not\in B^c$. Portanto, $\exists x | x \in A $ e $x \not\in B^c$. Logo, $A \not\subset B^c$. Uma contradição, pois temos como hipótese $A \subset B^c$. Deste modo, $A \subset B^c \Rightarrow A \cap B = \emptyset$.

    \paragraph{}
    De \textbf{1.} e \textbf{2.}, $A \cap B = \emptyset \Leftrightarrow A \subset B^c$.
    \end{proof}
    
    \paragraph{}
    Agora, vamos demonstrar que $A \cup B = E \Leftrightarrow A^c \subset B$.

    \begin{proof}
    Dividiremos a demonstração em duas partes: \textbf{1.} $A \cup B = E \Rightarrow A^c \subset B$ e \textbf{2.} $A^c \subset B \Rightarrow A \cup B = E$.

    \paragraph{1. $\Rightarrow$)}
    Suponhamos, por absurdo, que $A^c \not\subset B$. Neste caso, $\exists x| x \in A^c$ e $x \not\in B$. Como $x \not\in B \Leftrightarrow x \in B^c$, $\exists x| x \in A^c$ e $x \in B^c$. Assim, pela definição de interseção, $\exists x| x \in (A^c \cap B^c)$. Pela Lei de \textit{de Morgan}, $\exists x| x \in (A \cup B)^c$ e portanto, $\exists x | x \not\in (A \cup B)$. Uma contradição, pois $A \cup B = E$. Não pode existir um elemento que não esteja em $E$. Sendo assim, $A \cup B = E \Rightarrow A^c \subset B$.

    \paragraph{2. $\Leftarrow$)}
    Suponhamos, por absurdo, que $A \cup B \not= E$. Neste caso, $\exists x | x \not\in (A \cup B)$. Assim, $\exists x | x \in (A \cup B)^c$. Por \textit{de Morgan}, $\exists x | x \in (A^c \cap B^c)$. Pela definição de interseção, $\exists x | x \in A^c$ e $x \in B^c$. Como $x \in B^c \Leftrightarrow x \not\in B$, $\exists x | x \in A^c$ e $x \not\in B$. Logo, $A^c \not\subset B$. Uma contradição, pois $A^c \subset B$. Portanto, $A^c \subset B \Rightarrow A \cup B = E$.
    
    \paragraph{}
    De \textbf{1.} e \textbf{2.}, $A \cup B = E \Leftrightarrow A^c \subset B$.
    \end{proof}


    \subsection{Questão 4}
    \begin{proof}
    Dividiremos a demonstração em duas partes: \textbf{1.} $A \subset B \Rightarrow A \cap B^c = \emptyset$ e \textbf{2.} $A \cap B^c = \emptyset \Rightarrow A \subset B$.

    \paragraph{1. $\Rightarrow$)}
    Suponhamos, por absurdo, que $A \cap B^c \not= \emptyset$. Neste caso, $\exists x | x \in (A \cap B^c)$. Pela definição de interseção, $\exists x | x \in A $ e $x \in B^c$. Como $x \in B^c \Leftrightarrow x \not\in B$, então $\exists x | x \in A $ e $x \not\in B$. Portanto, $A \not\subset B$. Mas $A \not\subset B$ e $A \subset B$ (hipótese) não podem ser verdadeiros ao mesmo tempo. Um absurdo gerado pela nossa suposição inicial. Logo, $A \subset B \Rightarrow A \cap B^c = \emptyset$.

    \paragraph{2. $\Leftarrow$)}
    Suponhamos, por absurdo, que $A \not\subset B$. Neste caso, $\exists x | x \in A $ e $x \not\in B$. Assim, $\exists x | x \in A$ e $x \in B^c$. Pela definição de interseção, $\exists x | x \in (A \cap B^c)$. Mas não pode existir tal $x$, pois por hipótese, $A \cap B^c = \emptyset$. Portanto, $A \cap B^c = \emptyset \Rightarrow A \subset B$.
    
    \paragraph{}
    De \textbf{1.} e \textbf{2.}, $A \subset B \Leftrightarrow A \cap B^c = \emptyset$.
    \end{proof}


    \subsection{Questão 5}
    \paragraph{}
    Seja $A = \{1\}$, $B = \{2\}$ e $C = \{3\}$. Assim, $(A \cup B) \cap C = \emptyset$ e $A \cup(B \cap C) = \{1\}$. Logo, temos $(A \cup B) \cap C \not= A \cup (B \cap C)$.


    \subsection{Questão 6}
    \begin{proof}
    Dividiremos a demonstração de $X = A^c$ em duas etapas: \textbf{1.} $X \subset A^c$ e \textbf{2.} $A^c \subset X$.

    \paragraph{1.}
    Suponhamos, por absurdo, que $X \not\subset A^c$. Neste caso, $\exists x | x \in X $ e $x \not\in A^c$. Sendo assim, por definição de complementar, $\exists x | x \in X$ e $x \in A$. Por definição de interseção, $\exists x | x \in (X \cap A)$. Mas, por hipótese, $A \cap X = \emptyset$, não pode existir tal $x$. Da contradição segue que, de fato, $X \subset A^c$.

    \paragraph{2.}
    Suponhamos, por absurdo, que $A^c \not\subset X$. Neste caso, $\exists x | x \in A^c$ e $x \not\in X$. Por definição de complementar, $\exists x | x \in A^c$ e $x \in X^c$. Por definição de interseção, $\exists x | x \in (A^c \cap X^c)$. Por \textit{de Morgan}, $\exists x | x \in (A \cup B)^c$. E, novamente, pela definição de complementar, $\exists x | x \not\in (A \cup B)$. Uma contradição, pois parte da hipótese garante que $A \cup B = E$. Não pode existir tal x que não esteja em $E$. Assim, como o absurdo provém da nossa suposição inicial, concluímos que $A^c \subset X$.

    \paragraph{}
    De \textbf{1.} e \textbf{2.}, $X = A^c$.
    \end{proof}


    \subsection{Questão 7}
    Vamos provar inicialmente que $A \subset B \Rightarrow B \cap (A \cup C) = (B \cap C) \cup A, \forall C$.

    \begin{proof}
    Por distributividade, $B \cap (A \cup C) = (B \cap A) \cup (B \cap C)$. Como, por hipótese, $A \subset B$, então $B \cap A = A$. Assim, $(B \cap A) \cup (B \cap C) = A \cup (B \cap C)$. Por comutatividade, $A \cup (B \cap C) = (B \cap C) \cup A$. Como $C$ não possui nada em particular, temos que, $B \cap (A \cup C) = (B \cap C) \cup A$, $\forall C$.
    \end{proof}

    Agora, vamos provar que $\exists C | B \cap (A \cup C) = (B \cap C) \cup A \Rightarrow A \subset B$.
    \begin{proof}
    Se $B \cap (A \cup C) = (B \cap C) \cup A$, então $(B \cap C) \cup A \subset B \cap (A \cup C)$. Agora, consideremos um $x \in A$. Se $x \in A$, então $x \in (B \cap C) \cup A$. Mas, $(B \cap C) \cup A \subset B \cap (A \cup C)$, então, $x \in B \cap (A \cup C)$. Portanto, $x \in B$. Como todo $x$ em A também está em B, $A \subset B$.
    \end{proof}

    
    \subsection{Questão 8}
    \begin{proof}
    Dividiremos a demonstração em duas partes: \textbf{1.} $A = B \Rightarrow (A \cap B^c) \cup (A^c \cap B) = \emptyset$ e \textbf{2.} $(A \cap B^c) \cup (A^c \cap B) = \emptyset \Rightarrow A = B$.

    \paragraph{1. $\Rightarrow$)}
    Como $A = B$, então $(A \cap B^c) \cup (A^c \cap B) = (A \cap A^c) \cup (A^c \cap A)$. Mas, por comutatividade, $A \cap A^c = A^c \cap A$. Assim, $(A \cap A^c) \cup (A^c \cap A) = A \cap A^c$. Mas, nenhum elemento pode estar em um conjunto e em seu complementar ao mesmo tempo, portanto, $A \cap A^c = \emptyset$. Logo, $(A \cap B^c) \cup (A^c \cap B) = \emptyset$, quando $A = B$.

    \paragraph{2. $\Leftarrow$)}
    Suponhamos, por absurdo, que $A \not= B$. Assim, $A \not\subset B$ ou $B \not\subset A$.

    \paragraph{}
    No primeiro caso, $A \not\subset B \Rightarrow \exists x | x \in A$ e $x \not\in B$. Pela definição de complementar, $\exists x | x \in A$ e $x \in B^c$, e pela definição de interseção, $\exists x | x \in (A \cap B^c)$. Assim, $A \cap B^c \supset \{x\}$ e, por conseguinte, $(A \cap B^c) \cup (A^c \cap B) \supset \{x\}$. Logo, $(A \cap B^c) \cup (A^c \cap B) \not= \emptyset$. Uma contradição, pois, por hipótese, $(A \cap B^c) \cup (A^c \cap B) = \emptyset$. Portanto, $A \subset B$.

    \paragraph{}
    De modo análogo, no segundo caso, $B \not\subset A \Rightarrow \exists x | x \in B$ e $x \not\in A$. Pela definição de complementar, $\exists x | x \in B$ e $x \in A^c$, e pela definição de interseção, $\exists x | x \in (B \cap A^c)$. Assim, $B \cap A^c = A^c \cap B \supset \{x\}$ e, por conseguinte, $(A \cap B^c) \cup (A^c \cap B) \supset \{x\}$. Logo, $(A \cap B^c) \cup (A^c \cap B) \not= \emptyset$. Uma contradição, pois, por hipótese, $(A \cap B^c) \cup (A^c \cap B) = \emptyset$. Portanto, $B \subset A$.

    \paragraph{}
    Em ambos os casos, nossa suposição gerou absurdos. Logo, $(A \cap B^c) \cup (A^c \cap B) = \emptyset \Rightarrow A = B$.

    \paragraph{}
    Assim, de \textbf{1.} e \textbf{2.}, $A = B \Leftrightarrow (A \cap B^c) \cup (A^c \cap B) = \emptyset$.

    \end{proof}


    \subsection{Questão 9}
    \begin{proof}
    Dividiremos a demonstração em duas partes: \textbf{1.} $(A - B) \cup (B - A) \subset (A \cup B) - (A \cap B)$ e \textbf{2.} $(A \cup B) - (A \cap B) \subset (A - B) \cup (B - A)$.

    \paragraph{1.}
    Seja $x \in (A - B) \cup (B - A)$. Pela definição de união, sabemos que $x \in (A - B)$ ou $x \in (B - A)$. 
    
    \paragraph{}
    No primeiro caso, se $x \in (A - B)$, então $x \in A$ e $x \not\in B$. Mas, se $x \in A$, então, $x \in (A \cup B)$. Como $x \not\in B$, então $x \not\in (A \cap B)$. Assim, $x \in (A \cup B)$ e $x \not\in (A \cap B)$. Portanto, por definição de diferença, $x \in (A \cup B) - (A \cap B)$.

    \paragraph{}
    No segundo caso, se $x \in (B - A)$, então $x \in B$ e $x \not\in A$. Mas, se $x \in B$, então, $x \in (A \cup B)$. Como $x \not\in A$, então $x \not\in (A \cap B)$. Assim, $x \in (A \cup B)$ e $x \not\in (A \cap B)$. Portanto, por definição de diferença, $x \in (A \cup B) - (A \cap B)$.

    \paragraph{}
    Em qualquer caso, $(A - B) \cup (B - A) \subset (A \cup B) - (A \cap B)$.

    \paragraph{2.}
    Seja $x \in (A \cup B) - (A \cap B)$. Pela definição de diferença, $x \in (A \cup B)$ e $x \not\in (A \cap B)$. Se $x \in (A \cup B)$, então, pela definição de união, $x \in A $ ou $x \in B$. Vejamos cada caso. Se $x \in A$, então, $x \not\in B$, pois $x \not\in (A \cap B)$. Portanto, $x \in (A - B)$. Se $x \in B$, então, $x \not\in A$, pois $x \not\in (A \cap B)$. Portanto, $x \in (B - A)$. Em qualquer caso, $x \in (A - B) \cup (B - A)$. 

    \paragraph{}
    Portanto, em todo caso, $(A \cup B) - (A \cap B) \subset (A - B) \cup (B - A)$.
    
    \paragraph{}
    De \textbf{1.} e \textbf{2.}, $(A - B) \cup (B - A) = (A \cup B) - (A \cap B)$.
    \end{proof}


    \subsection{Questão 10}
    \begin{proof}
    Dividiremos a demonstração em duas partes: \textbf{1.} $A \Delta B = A \Delta C \Rightarrow B \subset C$ e \textbf{2.} $A \Delta B = A\Delta C \Rightarrow C \subset B$.

    \paragraph{1.}
    Seja $x \in B$. Assim, $x \in (B - A)$ ou $x \in (A \cap B)$.

    \paragraph{}
    No primeiro caso, se $x \in (B - A)$, então $x \in (A - B) \cup (B - A)$. Pela definição de diferença simétrica ($\Delta$), $x \in A \Delta B$. E, por hipótese, $x \in A \Delta C$. Assim, novamente pela definição de $\Delta$, $x \in (A - C) \cup (C - A)$. Mas, se $x \in (B - A)$, então, $x \not\in (A - C)$. Assim, $x \in (C - A)$ e finalmente, temos que $x \in C$.

    \paragraph{}
    No segundo caso, se $x \in (A \cap B)$, então, $x \not\in (A \cup B) - (A \cap B)$. Mas, utilizando o resultado da questão 9, $x \not\in (A - B) \cup (B - A)$. Assim, pela definição de $\Delta$, $x \not\in A \Delta B$, e, por hipótese, $x \not\in A \Delta C$. Novamente, pela definição de $\Delta$, $x \not\in (A - C) \cup (C - A)$. Portanto, $x \not\in (A - C)$ e $x \not\in (C - A)$. Se $x \not\in (A - C)$, então, $x \not\in A$ ou $x \in C$. Mas, $x \in A$, pois $x \in (A \cap B)$. Logo, $x \in C$.

    \paragraph{}
    Assim, em qualquer caso, $x \in B \Rightarrow x \in C$. Logo, $B \subset C$.

    \paragraph{2.}
    Análogo ao item \textbf{1.}. Basta intercambiar $B$ com $C$. Assim, concluímos que $C \subset B$.

    \paragraph{}
    De \textbf{1.} e \textbf{2.}, $A \Delta B = A \Delta C \Rightarrow B = C$.
    \end{proof}

    \paragraph{}
    Agora, vamos examinar a validade do resultado para $\cup$, $\cap$, e $\times$.

    \paragraph{$\cup$)}
    Vejamos um contra-exemplo para $A \cup B = A \cup C \Rightarrow B = C$. Seja $A = \{1, 2, 3\}$, $B = \{2\}$ e $C = \{3\}$. Assim, $A \cup B = A \cup C = \{1, 2, 3\}$, mas $B \not= C$.

    \paragraph{$\cap$)}
    Vejamos um contra-exemplo para $A \cap B = A \cap C \Rightarrow B = C$. Seja $A = \{1, 2\}$, $B = \{2, 3\}$ e $C = \{2, 4\}$. Assim, $A \cap B = A \cap C = \{2\}$, mas $B \not= C$.

    \paragraph{$\times$)}
    Seja $A = \emptyset$, $B = \{1\}$ e $C = \{2\}$, temos que $A \times B = A \times C = \emptyset$, mas $B \not= C$. Portanto, para $A = \emptyset$ não temos $A \times B = A \times C \Rightarrow B = C$. Vejamos o caso em que $A \not= \emptyset$.

    \begin{proof}
    Vamos demonstrar que para $A \not= \emptyset, A \times B = A \times C \Rightarrow B = C$. Para tanto, dividiremos a demonstração em duas partes: \textbf{1.} $A \times B = A \times C \Rightarrow B \subset C$ e \textbf{2.} $A \times B = A \times C \Rightarrow C \subset B$.

    \paragraph{1.}
    Suponhamos, por absurdo, que $B \not\subset C$. Neste caso, $\exists x | x \in B$ e $x \not\in C$. Como $A \not= \emptyset$, podemos tomar um $a \in A$ e formar um par $(a, x)$. Este par $(a, x) \in A \times B$, pois $a \in A$ e $x \in B$, mas não é verdade que $(a, x) \in A \times C$, pois $x \not\in C$. Assim, $A \times B \not= A \times C$, pois existe um elemento em $A \times B$ que não está em $A \times C$. Uma contradição com nossa hipótese. Segue, portanto, que $B \subset C$.

    \paragraph{2.}
    Suponhamos, por absurdo, que $C \not\subset B$. Neste caso, $\exists x | x \in C$ e $x \not\in B$. Como $A \not= \emptyset$, podemos tomar um $a \in A$ e formar um par $(a, x)$. Este par $(a, x) \in A \times C$, pois $a \in A$ e $x \in C$, mas não é verdade que $(a, x) \in A \times B$, pois $x \not\in B$. Assim, $A \times C \not= A \times B$, pois existe um elemento em $A \times C$ que não está em $A \times B$. Uma contradição com nossa hipótese. Segue, portanto, que $C \subset B$.

    \paragraph{}
    De \textbf{1.} e \textbf{2.}, para $A \not= \emptyset$, $A \times B = A \times C \Rightarrow B = C$.
    \end{proof}

    
    \subsection{Questão 11}
    \paragraph{a) $(A \cup B) \times C = (A \times C) \cup (B \times C)$}
    \begin{proof}
    Dividiremos a nossa demonstração em duas partes: \textbf{1.} $(A \cup B) \times C \subset (A \times C) \cup (B \times C)$ e \textbf{2.} $(A \times C) \cup (B \times C) \subset (A \cup B) \times C$.

    \paragraph{1.}
    Seja $(x, y) \in (A \cup B) \times C$. Por definição de produto cartesiano, $x \in A \cup B$ e $y \in C$. Assim, $x \in A$ ou $x \in B$. Se $x \in A$, então $(x, y) \in A \times C$. Se $x \in B$, então $(x, y) \in B \times C$. Assim, qualquer que seja o caso, $(x, y) \in (A \times C) \cup (B \times C)$. Portanto, $(A \cup B) \times C \subset (A \times C) \cup (B \times C)$.

    \paragraph{2.}
    Seja $(x, y) \in (A \times C) \cup (B \times C)$. Assim, por definição de união, $(x, y) \in A \times C$ ou $(x, y) \in B \times C$. Mas, pela definição de produto cartesiano, temos, $x \in A$ e $y \in C$ ou $x \in B$ e $y \in C$ e portanto, $x \in A$ ou $x \in B$ e $y \in C$. Assim, pela definição de união, $x \in (A \cup B)$ e $y \in C$. Logo, $(x, y) \in (A \cup B) \times C$. Assim, $(A \times C) \cup (B \times C) \subset (A \cup B) \times C$.
    
    \paragraph{}
    De \textbf{1.} e \textbf{2.}, $(A \cup B) \times C = (A \times C) \cup (B \times C)$.
    \end{proof}
    


\end{document}


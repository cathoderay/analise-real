\documentclass[11pt,a4paper]{report} 
\usepackage[portuges]{babel}
\usepackage[utf8]{inputenc}
\usepackage{graphicx}
\usepackage{amsthm,amssymb}
\usepackage{amsfonts}
\usepackage{indentfirst}
\newtheorem{lema}{letter}
\newtheorem{corolario}{Corolario}
\newtheorem{teorema}{Teorema}
\renewcommand\thesection{\arabic{section}}
\renewcommand{\qedsymbol}{\rule{0.7em}{0.7em}}
\graphicspath{{images//}}


\title{Soluções do livro "Curso de Análise" vol. 1}
\author{Ronald A. Kaiser}


\begin{document}
    \maketitle
    \newpage
    
    \paragraph{Notas}
    As soluções aqui apresentadas não contém seus respectivos enunciados. Algumas hipóteses fornecidas no livro não são enunciadas aqui, mas são utilizadas em algumas demonstrações. O leitor atento não deve ter dificuldades para acompanhar as soluções. Caso tenha alguma dúvida, recorra à obra original "Curso de análise" (vol. 1) de Elon Lages Lima.

    \section{Capítulo 1}

    \subsection{Questão 1}
    \begin{proof}
    Para demonstrarmos que $X = A \cup B$, provaremos que \textbf{1.} $X \subset (A \cup B)$ e \textbf{2.} $(A \cup B) \subset X$.

    \paragraph{1.}
    Sabemos que $\forall A$ e $\forall B$, $A \subset (A \cup B)$ e $B \subset (A \cup B)$. 
    Da 2$^a$ hipótese, $A \subset (A \cup B)$ e $B \subset (A \cup B) \Rightarrow X \subset (A \cup B)$.

    \paragraph{2.}
    A partir da 1$^a$ hipótese e da definição de inclusão, $x \in A \Rightarrow x \in X$ e $y \in B \Rightarrow y \in X$. Assim, todo elemento de A ou de B também pertence a X. Mais formalmente: $z \in A$ ou $z \in B \Rightarrow z \in X$, e  pela definição de união, $z \in (A \cup B) \Rightarrow z \in X$. Portanto, $(A \cup B) \subset X$.

    \paragraph{}
    De \textbf{1.} e \textbf{2.}, $X = (A \cup B)$.
    \end{proof}


    \subsection{Questão 2}
    \paragraph{Enunciado}
    Dados os conjuntos A e B, seja X um conjunto com as seguintes propriedades:

    1$^{a}$ $X \subset A$ e $X \subset B$

    2$^{a}$ $Y \subset A$ e $Y \subset B \Rightarrow Y \subset X$
    
    Prove que $X = A \cap B$.
    \begin{proof}
    Para demonstrarmos que $X = A \cap B$, provaremos que \textbf{1.} $X \subset (A \cap B)$ e \textbf{2.} $(A \cap B) \subset X$.

    \paragraph{1.}
    A partir da 1$^a$ hipótese e da definição de inclusão, segue que: $x \in X \Rightarrow x \in A$ e $x \in X \Rightarrow x \in B$. Assim, todo elemento de X pertence também aos conjuntos A e B. Portanto, $x \in X \Rightarrow x \in A$ e $x \in B$. Logo, pela definição de interseção, $x \in X \Rightarrow x \in (A\cap B)$. Donde concluímos que $X \subset (A \cap B)$.

    \paragraph{2.}
    Sabemos que $\forall A$ e $\forall B$, $(A \cap B) \subset A$ e $(A \cap B) \subset B$. Da 2$^{a}$ hipótese, temos que $(A \cap B) \subset A$ e $(A \cap B) \subset B \Rightarrow (A \cap B) \subset X$. Portanto, $(A \cap B) \subset X$.

    \paragraph{}
    De \textbf{1.} e \textbf{2.}, $X = (A \cap B)$.
    \end{proof}


    \subsection{Questão 3}
    \begin{proof}
    Provaremos inicialmente que $A \cap B = \emptyset \Leftrightarrow A \subset B^c$.

    \paragraph{1. $\Rightarrow)$}
    Suponhamos, por absurdo, que $A \not\subset B^c$. Neste caso, $\exists x | x \in A$ e $x \not\in B^c$. Mas, pela definição de complementar, $x \not\in B^c \Leftrightarrow x \in B$. Assim, $\exists x | x \in A$ e $x \in B \Rightarrow \exists x | x \in (A \cap B)$. Mas, por hipótese, $A \cap B = \emptyset$. Não pode existir tal $x$. Um absurdo gerado pela nossa suposição inicial. Portanto, $A \cap B = \emptyset \Rightarrow A \subset B^c$.

    \paragraph{2. $\Leftarrow)$}
    TODO

    \paragraph{}
    De \textbf{1.} e \textbf{2.}, $A \cap B = \emptyset \Leftrightarrow A \subset B^c$.
    \end{proof}
        
\end{document}

